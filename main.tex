\documentclass[a0]{tumposter}

\usepackage[english]{babel}

\usepackage{blindtext}

\usepackage{xcolor} 
\usepackage{multicol}
\usepackage{amsmath}
\usepackage{graphicx}
\usepackage{tabularx}

\usepackage[utf8]{inputenc}
\usepackage{booktabs}
\usepackage{natbib}
\usepackage{tikz}
\usetikzlibrary{arrows, positioning, 3d, calc, fit}
\usetikzlibrary{spy}
\usetikzlibrary{backgrounds}
\usetikzlibrary{matrix}

\usepackage{pgfplots}
\usepgfplotslibrary{groupplots}
\usepackage{pgfplotstable}
\usepgfplotslibrary{dateplot}

\newcommand{\pre}[1]{
	{\color{focusone}\textbf{#1}}
}
\newcommand{\post}[1]{
	{\color{focustwo}\textbf{#1}}
}


% for printing fontsizes
\usepackage{printlen}
\uselengthunit{mm}

\input{networks.tikz}
\input{recurrentcells.tikz}
\usepackage[utf8]{inputenc}

\pgfplotsset{every tick label/.append style={font=\tiny}}


\title{
	Convolutional LSTMs for Cloud-Robust \\ Segmentation of Remote Sensing Imagery%\\% 
	%{\Large \color{tumblue}  Multi-\{temporal, modal, resolution\} Data Fusion and Recurrent Nets}
	}
	
\author{
	Marc Rußwurm, Marco Körner
	}

\header{
	Remote Sensing Technology \\
	TUM Department of Civil, Geo and Environmental Engineering \\
	Technical University of Munich
	}
	
\begin{document}
\maketitle


\input{activations.tikz}
\newcounter{tcounter}

% provides \figourapproach and \figcloudfilteringpipeline
\input{images/overviewgraphs.tikz} 

%\vspace{1em}
\vspace{-1em}
\section{Cloud coverage: An omnipresent challenge to optical Earth observation models}

\begin{minipage}[t]{0.49\linewidth}
	
	\textbf{Optical Earth observation satellites}
	\begin{itemize}
		\item measure the reflected sunlight
		\item of surface objects that are
		\item often covered by clouds
		\item in a large number of spectral bands
	\end{itemize}
	
	%\Large\textbf{Finding clouds is rarely the final remote sensing objective}
	
\end{minipage}
\begin{minipage}[t]{0.49\linewidth}

	Cloud coverage over the 100km by 40km area of interest.
\input{scl.tikz}

\end{minipage}

\vspace{1em}
\begin{minipage}[t]{0.49\linewidth}
\section{Separate cloud pre-classification}

\textbf{Identifying  clouds in remote sensing}
\begin{itemize}
	\item is rarely the final objective
	\item is often performed by a separate cloud-classification model $f_{cl}$
	\item is usually a required preprocessing step for many approaches
\end{itemize}
%
%Many Remote Sensing approaches perform a cloud pre-classification.
%However, the pre-classification of clouds introduces a second model that is often treated as black-box.% \textbf{two separate models} \\ for cloud classification $f_{cl}$ and prediction $f$.

\vspace{.5em}
%\textbf{Complex pipeline of models}:
%\begin{center}
	\begin{tikzpicture}
		\node[draw, rounded corners, fill=tumbluelight!20, label=below:{\small complex processing pipeline}]{\figcloudfilteringpipeline};
	\end{tikzpicture}
	
%\end{center}
\vspace{.5em}

%\footnotemark[1] $\V{m} = f_cl(\V{x})$ to obtain a cloud mask $\V{m}$ that is used to filter the data (e.g., removal or temporal interpolation\footnotemark[2]) or masking the model-loss. \footnotemark.




%\node[noteannot, above right= 1em and -3.5em of approach](annottext){};
%
%\draw[annotation] (annottext) -- (approach);
%\draw[annotation] (annottext) -- (cloudclass);


%{\scriptsize
%	Roberto Interdonato, Dino Ienco, Raffaele Gaetano, and Kenji Ose. Duplo: A dual view point deep learning architecture for time series classification. arXiv preprint arXiv:1809.07589, 2018.
%}
%
%{\scriptsize
%	Zhiwei Li, Huanfeng Shen, Qing Cheng, Yuhao Liu, Shucheng You, and Zongyi He. Deep learning
%	based cloud detection for remote sensing images by the fusion of multi-scale convolutional features.
%	arXiv preprint arXiv:1810.05801, 2018.
%}

%\begin{tikzpicture}[node distance=2em]
%\node[process, label=below:{\tiny optical satellite images}](inputs){image(s) $\V{x}$};
%\node[process,right=7em of inputs, label=below:{\tiny prediction model}](approach){$\V{y}=f(\V{x}^\ast)$};
%\node[process, label=below:{\tiny interp.}](filter) at ($ (inputs)!0.5!(approach) $) {$f_i(\V{x},\V{m})$};
%\node[process, label={\tiny cloud pre-classification}](cloudclass) at ($ (inputs)!0.5!(approach) + (0,3em) $) {$f_{cl}(\V{x})$};
%
%\draw[conn] (inputs) -- (cloudclass);
%\draw[conn] (inputs) -- node[near end, above]{$\V{x}$}(filter);
%\draw[conn] (filter) -- node[near end, above]{$\V{x}^\ast$}(approach);
%\draw[conn] (cloudclass) -- node[near end, left]{$\V{m}$}(filter);
%\end{tikzpicture}

\end{minipage}
\hfill
\begin{minipage}[t]{0.49\linewidth}
\section{Learn to ignore clouds in one model}

\textbf{In this work}
\begin{itemize}
	\item We treat clouds as data-inherent noise
	\item by employing ConvLSTMs to learn cloud masking
	\item and classification in \textbf{\color{tumbluedark}one model end-to-end}.
\end{itemize}

\vspace{.7em}
%\textbf{Straightforward end-to-end model training}:
%\vspace{1em}
%\begin{center}
	\begin{tikzpicture}
		\node[draw, rounded corners, fill=tumbluelight!20, label=below:{\small our straightforward end-to-end trainable model}, inner ysep=1em]{\figourapproach};
	\end{tikzpicture}
	
%\end{center}
%\vspace{2em}

%In prior work we introduced sequential satellite imagery to a ConvLSTM model without prior cloud filtering and achieved state-of-the-art classification accuracies.
%



%{\tiny
%Marc Rußwurm and Marco Körner. Multi-temporal land cover classification with sequential recurrent
%encoders. ISPRS International Journal of Geo-Information, 7(4):129, 2018.
%}
\end{minipage} 


\section{The cloud-robust ConvLSTM model for vegetation classification}

\begin{minipage}{0.49\linewidth}
	\input{seqencnetwork.tikz}
	\figseqencnetwork
\end{minipage}
\hspace{1em}
\begin{minipage}{0.49\linewidth}
	
\vspace{.5em}
\textbf{Our ConvLSTM model} $f(\V{X})$
\begin{itemize}
	\item encodes sequence of images $\V{x}$ is by a ConvLSTM
	\item produces class activations $\V{y}$ by final conv+softmax.
\end{itemize}
 
\vspace{1em}
\textbf{Trained to }
\begin{itemize}
	\item classify the vegetation type (maize, wheat, barley etc.)
	\item using 80k image sequences of 46 observations.
	
\end{itemize}
\vspace{1em}

\textbf{We achieved}
\begin{itemize}
	\item state-of-the-art accuracies in crop classification
	\item without prior cloud filtering
	\item published results in a remote sensing related journal$^1$.
\end{itemize}

\end{minipage}

\vspace{1em}
{\small $^1$Marc Rußwurm and Marco Körner. Multi-temporal land cover classification with sequential recurrent
	encoders. ISPRS International Journal of Geo-Information, 2018.}
%\begin{tikzpicture}
%	\node[draw, inner sep=.5em, rounded corners](network){};
%	\node[left=2.5em of network.north west, noteannot, text width=9em, anchor=north east]{
%		\textbf{In prior work} \\ we trained a ConvLSTM for field crop identification.
%		We use this recurrent network to on a \textbf{sequence of satellite images} to approximate \textbf{vegetation life cycle} events.};
%	\node[right=2.5em of network.north east, anchor= north west, noteannot, text width=9em]{
%		We insert a \pre{sequence of satellite images} to learn \textbf{vegetation life cycle} events for field crop identification.};
%	
%\end{tikzpicture}

\vspace{.5em}

\section{Did the ConvLSTM learn to ignore clouds?}

%We conducted two experiments to explore how clouds have been processed within the network. 
%\begin{tikzpicture}
%	\node[rounded corners, draw=none, text=tumblack, inner sep=1em](a){\LARGE \textbf{Did the ConvLSTM learn to ignore clouds?}};
%\end{tikzpicture}
%\vspace{1em}
%\input{examples.tex}

\begin{minipage}{0.49\linewidth}
	%\section{Varying cloud coverage and observe accuracy}		
	%\vspace{.2em}
	We \textbf{trained} the network on \textbf{cloudy} and \textbf{non-cloudy} datasets \\
	observing the validation accuracy:
	
	\vspace{1em}
	
	\input{clouds.tikz}
	
	%Robust classification on different cloud coverages
	
\end{minipage}
\hfill
\begin{minipage}{0.49\linewidth}
	
	We \textbf{visualized} some \textbf{hidden LSTM states}
	
	\vspace{1em}
	
	%\section{Visualize and explore hidden states}	
	\includegraphics[width=\textwidth]{images/activations}
	
	\vspace{1em}
	
	%Some input $\V{i}$ and modulation gates $\V{j}$ approach zero on clouds. 
%	\begin{tikzpicture}
%		\node[](actnormal){\figactivations{47}};
%		
%		\node[noteannot, above= 0em of actnormal, xshift=5em](a){\textbf{cells extract features} sequentially from the input data};
%		\draw[annotation] (a)++(5em,-1em) -- ++(2em,-5em);
%		\draw[annotation] (a)++(5em,-1em) -- ++(2em,-7em);
%		
%		\node[noteannot, below= .5em of actnormal, xshift=-5em](b){\textbf{cells mask clouds} indicated by the input gate $\V{i}$ closing (values of zero)};
%		\draw[annotation] (b) -- ++(-4em,2em);
%		\draw[annotation] (b) -- ++(10em,2em);
%		\draw[annotation] (b) -- ++(21em,2em);
%		
%		\node[left=3em of a]{LSTM cell activations:};
%		
%	\end{tikzpicture}
%	
\end{minipage} 

\vspace{1em}


% footer environment places \hfill and sets fontsize
\begin{footer}
	\begin{multicols}{2}
		\textbf{Technical University of Munich}\\
		TUM Department of Civil, Geo and Environmental Engineering \\
		Chair of Remote Sensing Technology, Computer Vision Research Group \\
		Arcisstr. 21, 80333 Munich, Germany \\
		www.lmf.bgu.tum.de/vision
		\vfill\columnbreak
		%right
		\begin{multicols}{2}
			\textbf{Authors} \\
			Marc Rußwurm \\ (marc.russwurm@tum.de) \\
			Marco Körner \\ (marco.koerner@tum.de)
			\vfill
			\columnbreak
			\textbf{Code \& Data} \\
			github.com/TUM-LMF/MTLCC \\
			github.com/TUM-LMF/MTLCC-pytorch \\
			\\
			twitter.com/MarcCoru
			\vfill
%			https://github.com/TUM-LMF/MTLCC
%			\hfill\includegraphics[height=5.5cm]{images/qrcode.png}
			\fancywhitespace		
		\end{multicols}
		
	\end{multicols}
\end{footer}


\end{document}